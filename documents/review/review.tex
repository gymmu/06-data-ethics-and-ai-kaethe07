\documentclass{article}

\usepackage[ngerman]{babel}
\usepackage[utf8]{inputenc}
\usepackage[T1]{fontenc}
\usepackage{hyperref}
\usepackage{csquotes}
\usepackage {geometry}

\usepackage[
    backend=biber,
    style=apa,
    sortlocale=de_DE,
    natbib=true,
    url=true,
    doi=false,
    sortcites=true,
    sorting=nyt,
    isbn=false,
    hyperref=true,
    backref=false,
    giveninits=false,
    eprint=false]{biblatex}
\addbibresource{../references/bibliography.bib}

\title{Review des Papers "Welche ethischen Herausforderungen entstehen durch 
den Einsatz von KI im Gesundheitswesen?" von Lars Burkard}
\author{Käthe Linke Aguirre}
\date{\today}

\begin{document}

\maketitle

\newpage

\abstract{
In diesem Dokument wird über den Einsatz von Künstlicher Intelligenz (KI) im Gesundheitswesen
gesprochen und welche ethischen Herausforderungen damit verbunden sind. 
\begin{itemize}
    
\item \underline{Kapitel 1: 1.1 Was ist KI? und 1.2 Wie wird KI trainiert?}
\\
Das Dokument beginnt mit einer Einleitung, in der erklärt wird, was Künstliche Intelligenz (KI)
ist und wie sie trainiert wird. Die Einleitung bietet eine klare Übersicht über die 
Grundlagen der KI. John McCarthys Definition wird im Text sinnvoll hervorgehoben, 
und es wird  erklärt, dass KI Sensoren, Datenverarbeitung und Aktoren nutzt. Besonders gut wurde der Unterschied 
zwischen schwacher KI und starker KI und ihre Funktionsweise erklärt. Allerdings könnten die ethischen Herausforderungen 
der starken KI genauer beschrieben werden, um den Text verständlicher zu machen.
Es wäre auch interessant gewesen einige Beispiele zur starken und schwachen KI anzugeben. Date (Star Trek) und Terminator wären
z.B. Beispiele für starke KI und ChatGPT, der Tesla-Autopilot und Google für schwache KI. \citep{HubSpot}
\\
Der Abschnitt über das Training von KI ist informativ und gut strukturiert. 
Die drei Teilschritte - Training, Validierung und Testen - werden gut beschrieben. 
Die Erklärung von maschinellem Lernen und Deeplearning ist verständlich und auch die 
Unterschiede zwischen dem überwachtem und unüberwachtem Lernen. 
Allerdings hätte man auch einige Beispiele zum überwachten und unüberwachten Lernen angegeben können, um zu zeigen, 
in welchen Fällen sie angewendet werden. Das unüberwachte maschinelle Lernen wird zum Beispiel verwendet, um Risikofaktoren
für Krankheiten identifizieren zu können und Präventionsmassnahmen zu planen. Das unüberwachte maschinelle Lernen wird 
hingegen verwendet, um zum Beispiel Personen nach unterschiedlichen Interessen klassifizieren oder Kunden nach ihrem Kaufverhalten 
zu gruppieren. \citep{capterra}

\item \underline{Kapitel 2: 2.1 Einbindung in Heilungsprozesse und was bewirkt es und 2.2 Fazit}

Der Abschnitt <<Einbindung in Heilungsprozesse und was bewirkt es>> zeigt die Vorteile von KI
im Gesundheitswesen. Die Verbesserung der Diagnosen, die effizientere Behandlung und die Einsparungen
durch Früherkennung von Krankheiten wie Krebs und Demenz werden gut dargestellt. Es wird ebenfalls die Bedeutung starker 
Führungskräfte und kontinuierlicher Weiterbildung des Pflegepersonals betont, was die Herausforderungen bei der Umsetzung
von KI im Gesundheitswesen verdeutlicht. Allerdings gibt es paar Aspekte, die in diesem Text nicht erwähnt wurden.
Man hätte verdeutlichen können, was KI benötigt, um die Diagnosen von Patienten zu verbessern. Damit sie präzise Diagnose stellen kann, 
braucht KI grosse Mengen an Daten, leistungsfähige Algorithmen und regelmässige Updates. Dadurch kann das medizinische Fachpersonal im Gesundheitswesen
effizienter arbeiten und schnellere Diagnosen erstellen. Zudem wäre es wichtig zu betonen, wie wichtig und entscheidend die Überprüfung von KI-erstellten Diagnosen
durch das medizinische Fachpersonal ist, um Fehldiagnosen zu vermeiden.  
Darüber hinaus, hätte man im Text auch einige konkrete Beispiele oder Fallstudien anführen können, die die Vorteile von KI im Gesundheitswesen
veranschaulichen. Ein anderer wichtiger Aspekt, den man hinzugefügt haben könnte, ist dass KI nicht nur die Diagnose und Behandlung verbessern kann, sondern
auch zu einer personalisierten Medizin beiträgt. KI ist in der Lage, aus grossen Mengen an Patientendaten individualisierte Behandlungspläne zu erstellen,
die den spezifischen Bedürfnissen des Patienten entsprechen. \citep{clutch}
Der Fazit fasst die Erkenntnisse vom Dokument gut zusammen und betont die Bedeutung einer verantwortungsvollen der KI im Gesundheitswesen. Die Einhaltung von ethischen Richtlinien und die Förderung
 von eine ganzheitliche Herangehensweise wird dabei hervorgehoben. 


\end{itemize}

\newpage

Insgesamt bietet der Text einen guten Überblick über die Bedeutung von KI im Gesundheitswesen, sowie die damit verbundenen Chancen und Herausforderungen.
Hätte man im Text weitere Beispiele und eine vertieft Diskussion über ethische Aspekte eingefügt, dann hätte der Text weiter verbessert werden können. 




}





\printbibliography

\end{document}
