\documentclass{article}

\usepackage[ngerman]{babel}
\usepackage[utf8]{inputenc}
\usepackage[T1]{fontenc}
\usepackage{hyperref}
\usepackage{csquotes}
\usepackage {geometry}

\usepackage[
    backend=biber,
    style=apa,
    sortlocale=de_DE,
    natbib=true,
    url=true,
    doi=false,
    sortcites=true,
    sorting=nyt,
    isbn=false,
    hyperref=true,
    backref=false,
    giveninits=false,
    eprint=false]{biblatex}
\addbibresource{../references/bibliography.bib}

\title{Review des Papers "Welche ethischen Herausforderungen entstehen durch 
den Einsatz von KI im Gesundheitswesen?" von Lars Burkard}
\author{Käthe Linke Aguirre}
\date{\today}

\begin{document}

\maketitle

\newpage

\abstract{
In diesem Dokument wird über den Einsatz von Künstlicher Intelligenz (KI) im Gesundheitswesen
geredet und welche ethischen Herausforderungen damit verbunden sind. 
\begin{itemize}
    
\item \underline{Kapitel 1: 1.1 Was ist KI? und 1.2 Wie wird KI trainiert?}
\\
Das Dokument beginnt mit eine Einleitung, in der erklärt wird, was Künstliche Intelligenz (KI)
ist und wie sie trainiert wird. Die Einleitung bietet eine klare Übersicht über die 
Grundlagen der KI. John McCarthys Definition wird im Text sinnvoll hervorgehoben, 
und es wird  erklärt, dass KI Sensoren, Datenverarbeitung und Aktoren nutzt. Besonders gut wurde der Unterschied 
zwischen schwacher KI und starker KI und ihre Funktionsweise erklärt. Allerdings könnte man die ethischen Herausforderungen 
der starken KI genauer beschrieben werden, um den Text verständlicher zu machen.
Es wäre auch interessant gewesen einige Beispiele zur starken und schwachen KI anzugeben. Date (Star Trek) und Terminator wären
z.B. Beispiele für starke KI und ChatGPT, der Tesla-Autopilot und Google für schwache KI. \citep{HubSpot}
\\

\item \underline{Kapitel 2: 2.1 Einbindung in Heilungsprozesse und was bewirkt es 
und 2.2 Fazit}

\end{itemize}




}





\printbibliography

\end{document}
