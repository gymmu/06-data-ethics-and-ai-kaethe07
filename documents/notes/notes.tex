\documentclass{article}

\usepackage[ngerman]{babel}
\usepackage[utf8]{inputenc}
\usepackage[T1]{fontenc}
\usepackage{hyperref}
\usepackage{csquotes}

\usepackage[
    backend=biber,
    style=apa,
    sortlocale=de_DE,
    natbib=true,
    url=false,
    doi=false,
    sortcites=true,
    sorting=nyt,
    isbn=false,
    hyperref=true,
    backref=false,
    giveninits=false,
    eprint=false]{biblatex}
\addbibresource{../references/bibliography.bib}

\title{Notizen zu KI in der Medizin}
\author{Käthe Linke Aguirre}
\date{\today}

\begin{document}
\maketitle

\abstract{
    Dieses Dokument ist eine Sammlung von Notizen zu dem Projekt. Die Struktur innerhalb des
    Projektes ist gleich ausgelegt wie in der Hauptarbeit, somit kann hier einfach geschrieben
    werden, und die Teile die man verwenden möchte, kann man direkt in die Hauptdatei ziehen.}

    \tableofcontents

\section { Einführung: Was ist KI? }

   
Künstliche Intelligenz (KI) ist die Fähigkeit einer Maschine, menschliche Fähigkeiten wie
wie logisches Denken, Lernen, Planen und Kreativität zu imitieren. KI ist einer der 
großen technologischen Fortschritte unserer Zeit. Ihre Entwicklung schreitet 
rasant voran und die Einsatzmöglichkeiten nehmen zu. KI ist ein wichtiger Bestandteil nicht nur in der Informatik,
sondern auch vieler anderer Bereiche unseres Lebens. KI wird z.B. in der Bildung verwendet, indem sie personalisierte
Lernsysteme erstellt oder die Rolle von Tutoren spielen können. Sie wird auch in der Kunst 
und Architektur verwendet, um z.B. das Bild eines Objekts zu erstellen.
Auch in der Medizin spielt KI eine bedeutende Rolle. In meinem Dokument werde ich mich
mit dem Einsatz von Künstlicher Intelligenz beshäftigen und ob dieser Einsatz
ethisch vertretbar ist oder nicht. 


\section{Wie wird KI in der Medizin verwendet?}


\section{Inwieweit beeinflusst der Einsatz von KI medizinischen
 Fachkräften bei Entscheidungen?}

\section {}






\input{section_ai.tex}

\printbibliography

\end{document}
