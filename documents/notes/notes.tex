\documentclass{article}

\usepackage[ngerman]{babel}
\usepackage[utf8]{inputenc}
\usepackage[T1]{fontenc}
\usepackage{hyperref}
\usepackage{csquotes}

\usepackage[
    backend=biber,
    style=apa,
    sortlocale=de_DE,
    natbib=true,
    url=false,
    doi=false,
    sortcites=true,
    sorting=nyt,
    isbn=false,
    hyperref=true,
    backref=false,
    giveninits=false,
    eprint=false]{biblatex}
\addbibresource{../references/bibliography.bib}

\title{Notizen zu KI in der Medizin}
\author{Käthe Linke Aguirre}
\date{\today}

\begin{document}
\maketitle

\abstract{
    Dieses Dokument ist eine Sammlung von Notizen zu dem Projekt. Die Struktur innerhalb des
    Projektes ist gleich ausgelegt wie in der Hauptarbeit, somit kann hier einfach geschrieben
    werden, und die Teile die man verwenden möchte, kann man direkt in die Hauptdatei ziehen.}

    \tableofcontents

\section { Was ist KI? }

\section {
    -KI steht für Künstliche Intelligenz
    -KI ist die Fähigkeit einer Maschine, menschliche Fähigkeiten wie logisches Denken, Lernen, Planen 
    und Kreativität zu imitieren. 
    -KI ist einer der großen technologischen Fortschritte unserer Zeit. Ihre Entwicklung schreitet 
    rasant voran und die Einsatzmöglichkeiten nehmen zu.
    -wichtiger Bestandteil nicht nur der Informatik, sondern auch in 
    viele anderen Bereichen. 
    -KI wird z.B. in der Bildung, Kunst, Architektur und Medizin verwendet. 
    -Auch ein wichtiger Bestandteil der Medizin. 
    -In meinem Dokument werde ich mich mit dem Einsatz von Künstlicher Intelligenz 
    in der Medizin beschäftigen und ob es ethisch vertretbar ist oder nicht. 

} 


\input{section_ai.tex}

\printbibliography

\end{document}
